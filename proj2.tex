% Project: ITY Projekt 2
% Autor:   Evgenii Shiliaev
% Datum:   16.03.2021

\documentclass[a4paper, twocolumn, 11pt]{article}
\usepackage[left=1.5cm, text={18cm, 25cm}, top=2.5cm]{geometry}
\usepackage[utf8]{inputenc}
\usepackage[IL2]{fontenc}
\usepackage[czech]{babel}
\usepackage{amsmath}
\usepackage{amsthm}
\usepackage{amssymb}
\usepackage{times}

\newtheorem{definition}{Definice}
\newtheorem{theorem}{Věta}


\begin{document}

\begin{titlepage}
	\begin{center}
		\Huge
		\textsc{Fakulta informačních technologií\\Vysoké učení technické v~Brně}

		\vspace{\stretch{0.382}}

		\LARGE
		Typografie a publikování – 2. projekt\\
		Sazba dokumentů a matematických výrazů

		\vspace{\stretch{0.618}}

		\Large
		2021 \hfill Evgenii Shiliaev (xshili00)

	\end{center}
\end{titlepage}

\section*{Úvod}

V~této úloze si vyzkoušíme sazbu titulní strany, matematic-kých vzorců, prostředí a dalších textových struktur obvyk-lých pro technicky zaměřené texty (například rovnice~\eqref{equation1} nebo Definice 1 na straně~\pageref{definice1})
Rovněž si vyzkoušíme pou-žívání odkazů \verb|\ref| a \verb|\pageref|.

Na titulní straně je využito sázení nadpisu podle op-tického středu s~využitím zlatého řezu.
Tento postup byl probírán na přednášce.
Dále je použito odřádkování se zadanou relativní velikostí 0.4 em a 0.3 em.

V~případě, že budete potřebovat vyjádřit matematickou konstrukci nebo symbol a nebude se Vám dařit jej nalézt v~samotném \LaTeX u, doporučuji prostudovat možnosti balíku maker \AmS-\LaTeX.

\section{Matematický text}
Nejprve se podíváme na sázení matematických symbolů a~výrazů v~plynulém textu včetně sazby definic a~vět s~využitím balíku \verb|amsthm|
Rovněž použijeme poznámku pod čarou s~použitím příkazu \verb|\footnote|.
Někdy je vhodné použít konstrukci \verb|\mbox{}|, která říká, že text nemá být zalomen.

\begin{definition}\label{definice1}
	\emph{Rozšířený zásobníkový automat} (RZA) je definován jako sedmice tvaru
	$A = (Q,\Sigma,\Gamma,\delta,q_0,Z_0,F)$, kde:

	\begin{itemize}

		\item[$\bullet$] Q je konečná množina \emph{vnitřních (řídicích) stavů,}
		\item[$\bullet$] $\Sigma$ je konečná \emph{vstupní abeceda,}
		\item[$\bullet$] $\Gamma$ je konečná \emph{zásobníková abeceda,}
		\item[$\bullet$] $\delta$ je \emph{přechodová funkce} $Q\times(\Sigma\cup\{\epsilon\})\times\Gamma^\ast \rightarrow2^{Q\times\Gamma^\ast}$,
		\item[$\bullet$] $q_0\in Q$ je \emph{počáteční stav}, $Z_0\in \Gamma$ je \emph{startovací symbol zásobníku} a $F \subseteq Q$ je množina \emph{koncových stavů.}

	\end{itemize}

	\emph{Nechť $P=(Q,\Sigma,\Gamma,\delta,q_0,Z_0,F)$ je rozšířený zásobníkový automat.} Konfigurací \emph{nazveme trojici $(q,w,\alpha) \in Q \times \Sigma^\ast \times \Gamma^\ast$, kde $q$ je aktuální stav vnitřního řízení, $w$ je dosud nezpracovaná část vstupního řetězce a $\alpha = Z_{i_{1}}Z_{i_{2}}\dots Z_{i_{k}}$ je obsah zásobníku\footnote{$Z_{i_{1}}$ je vrchol zásobníku}.}

\end{definition}


\subsection{Podsekce obsahující větu a odkaz}

\begin{definition}\label{definice2}
	\emph{Řetězec $w$ nad abecedou $\Sigma$ je přijat RZA}

	\noindent A~jestliže $(q_0,w,Z_0)\overset{*}{\underset{A}\vdash} (q_F,\epsilon,\gamma)$ pro nějaké $\gamma \in \Gamma^\ast$ a $q_F \in F$. $\text{Množinu } L(A)\; = \; \{w \mid w \text{ je přijat RZA A}\} \subseteq \Sigma^\ast$ nazýváme \emph{jazyk přijímaný RZA} A.

\end{definition}

Nyní si vyzkoušíme sazbu vět a důkazů opět s~použitím balíku \verb|amsthm|.

\begin{theorem}
	Třída jazyků, které jsou přijímány ZA, odpovídá \emph{bezkontextovým jazykům.}
\end{theorem}

\begin{proof}
	V~důkaze vyjdeme z~Definice~\ref{definice1} a~\ref{definice2}.
\end{proof}

\section{Rovnice a odkazy}
Složitější matematické formulace sázíme mimo plynulý text.
Lze umístit několik výrazů na jeden řádek, ale pak je třeba tyto vhodně oddělit, například příkazem \verb|\quad|.

$$\sqrt[i]{x_i^3} \text{\quad}\text{kde } x_i \text{ je } i\text{-té sudé číslo splňující\quad} x_i^{x_i^{i^2} + 2} \leq y_i^{x_i^4}$$

V~rovnici~\eqref{equation1} jsou využity tři typy závorek s~různou explicitně definovanou velikostí.

\begin{eqnarray}
	x & = & \bigg[ \Big\{ \left[ a + b \right] * c \Big\} ^d \oplus 2 \bigg]^{3/2}\label{equation1}\\
	y & = &\lim_{x \rightarrow \infty} \frac{\frac{1}{\log_{10} x }}{\sin^2 x + \cos^2 x} \nonumber
\end{eqnarray}

V~této větě vidíme, jak vypadá implicitní vysázení limity $\lim_{n\rightarrow\infty} f(n)$ v~normálním odstavci textu.
Podobně je to i s~dalšími symboly jako $\prod^n_{i=1} 2^i$ či $\bigcap_{A\in \mathcal{B}} A$.
V~případě vzorců $\lim\limits_{n \rightarrow \infty} f(n)$ a $\prod\limits^n_{i=1} 2^i$ jsme si vynutili méně úspornou sazbu příkazem \verb|\limits|.

\begin{equation}
	\int_b^a g(x)\, \mathrm{d}x = -\int\limits_a^b f(x)\, \mathrm{d}x
\end{equation}

\section{Matice}
Pro sázení matic se velmi často používá prostředí \verb|array| a závorky (\verb|\left|, \verb|\right|).

$$ \left (
	\begin{array}{ccc}
			a - b            & \widehat{\xi+\omega}    & \pi         \\
			\vec{\mathbf{a}} & \overleftrightarrow{AC} & \hat{\beta}
		\end{array}
	\right) = 1 \Longleftrightarrow \mathcal{Q} = \mathbb{R}$$

$$\mathbf{A} = \left \|
	\begin{array}{cccc}
		a_{11} & a_{12} & \dots  & a_{1n} \\
		a_{21} & a_{22} & \dots  & a_{2n} \\
		\vdots & \vdots & \ddots & \vdots \\
		a_{m1} & a_{m2} & \dots  & a_{mn}
	\end{array}
	\right \| = \left |
	\begin{array}{c l}
		t  & u~ \\
		v~ & w
	\end{array}
	\right| = tw - uv$$

Prostředí \verb|array| lze úspěšně využít i jinde.

$$ \binom{n}{k} = \left \{
	\begin{array}{c l}
		0                   & \text{pro } k~< 0 \text{ nebo } k~> n \\
		\frac{n!}{k!(n-k)!} & \text{pro } 0 \leq k~\leq n.
	\end{array}
	\right. $$

\end{document}
